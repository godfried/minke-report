\subsection{GAE}
In order to understand how the system will function it is important to get an idea of how and what data will be stored.
Our entities within the Datastore are the following:
\begin{description}
\item [ProductCategory] Three nested categories associated with a product, e.g. Coffee Beans, Coffee, Consumables.
\item [Brand] Brand associated with a product, e.g. Jacobs.
\item[Product] A specific product, e.g. 500g Frisco.
\item [City] A city or town, e.g. Stellenbosch, Western Cape, South Africa.
\item [CityLocation] A GPS location together with its closest city.
\item [Store] A store franchise, e.g. Spar.
\item [Branch] A branch of a Store, e.g. Eikestad Woolworths.
\item [BranchProduct] The entity associated with a product at a specific branch.
\item [DatePrice] The price associated with a BranchProduct on a particular Date.
\end{description}
Location coordinates will be stored using a convenience class called GPSCoords.
A clearer picture of the system is given in Figure 12.
\begin{center}
\begin{figure}[H]
\begin{tikzpicture}[node distance=4cm]
    \node (bp) [entity, rectangle split, rectangle split parts=2]
        {
            \textbf{BranchProduct}\\
            \nodepart{second}ID (PK)\\
            ProductID (FK)\\
            BranchID (FK)\\
            DatePriceID (FK)\\
        };
         \node (dp) [entity, rectangle split, rectangle split parts=2, right=1.5cm of bp]
        {
            \textbf{DatePrice} \\
            \nodepart{second}ID (PK)\\
			  BranchProductID (FK)\\
			  Date (Date)\\
			  Price (Integer)\\
        };    
         \node (branch) [entity, rectangle split, rectangle split parts=2, below of= dp]
        {
            \textbf{Branch} \\
            \nodepart{second}ID (PK)\\
            Name (String)\\
			StoreID (FK)\\
			CityLocationID (FK)\\
        };
        \node (product) [entity, rectangle split, rectangle split parts=2, left=1.5cm of bp]
        {
            \textbf{Product} \\
            \nodepart{second}ID (PK)\\
            Name (String)\\
			      BrandID (FK) \\
			      ProductCategoryID (FK)\\
			      Measurement (String)\\
        };
        \node (brand) [entity, rectangle split, rectangle split parts=2, below right=1.5cm and 1.5cm of product]
        {
            \textbf{Brand} \\
            \nodepart{second}ID (PK)\\
            Name (String)\\
        };
        \node (store) [entity, rectangle split, rectangle split parts=2, below of=brand ]
        {
            \textbf{Store} \\
            \nodepart{second}ID (PK)\\
            Name (String)\\
        };

        \node (loc) [entity, rectangle split, rectangle split parts=2, below of= branch]
        {
            \textbf{CityLocation} \\
            \nodepart{second}ID (PK)\\
            CityID (FK)\\
            Latitude\\
            Longitude\\
        };
        \node (pc) [entity, rectangle split, rectangle split parts=2, below of= product]
        {
            \textbf{ProductCategory} \\
            \nodepart{second}ID (PK)\\
            ProductID (FK)\\
            CategoryID(FK)\\
        };
        \node (category) [entity, rectangle split, rectangle split parts=2, below of= pc]
        {
            \textbf{Category} \\
            \nodepart{second}ID (PK)\\
            Name (String)\\
            ProductCategoryID (FK)\\
        };
         \node (city) [entity, rectangle split, rectangle split parts=2, below of= loc]
        {
            \textbf{City} \\
            \nodepart{second}ID (PK)\\
            Name (String)\\
            ProvinceID (FK)\\
    		    Latitude\\
            Longitude\\
    	};
    	\node (province) [entity, rectangle split, rectangle split parts=2, left=1.5cm of city]
        {
            \textbf{Province} \\
            \nodepart{second}ID (PK)\\
            Name (String)\\
            CountryID (FK)\\
    	};
    	\node (country) [entity, rectangle split, rectangle split parts=2, left=1.5cm of province]
        {
            \textbf{Country} \\
            \nodepart{second}ID (PK)\\
            Name (String)\\
    	};
     \path[->] (bp) edge node[above] {$1 \ldots *$} (dp);
     \path[->] (bp) edge node[above] {$1 \ldots *$} (product);
     \path[->] (bp) edge node[above, rotate=-45] {$* \ldots 1$} (branch);
	 \path[->] (product) edge node[above, rotate=-90] {$1 \ldots *$} (pc);
	 \path[->] (product) edge node[above, rotate=-45] {$* \ldots 1$} (brand);
	 \path[->] (branch) edge node[above, rotate=45] {$1 \ldots *$} (store);
	 \path[->] (branch) edge node[above, rotate=-90] {$* \ldots 1$} (loc);
     \path[->] (loc) edge node[above, rotate=-90] {$* \ldots 1$} (city);
    \path[->] (city) edge node[above] {$1 \ldots *$} (province);
  \path[->] (province) edge node[above] {$1 \ldots *$} (country);
    \path[->] (category) edge node[above, rotate=-90] {$* \ldots 1$} (pc);
 \end{tikzpicture}
 \caption{Class Diagram of Entities used.}
 \end{figure}
\end{center}

The entities all have their own Data Access Objects (DAOs) which allow for querying, get and put operations. The DAOs are 
static objects which provide a simple interface for interacting with the Objectify-Appengine Datastore. They are all created when the system is loaded. \\
Each of the services used by the GWT front-end has an implementation on the GAE back-end. These are:
\begin{description}
\item[ClassService] Registers entities with Objectify. Was also used to add mock data to the Datastore.
\item[BranchProductService] Retrieves the following:
\begin{itemize}
\item BranchProducts which match a ProductCategory and/or a City or a Product.
\item Branches which stock a set of Products.
\item BranchProducts's price histories.
\end{itemize}
\item[ProductService] Retrieves all Products from the Datastore.
\item[ProductCategoryService] Retrieves all ProductCategories from the Datastore.
\item[LocationService] Retrieves all cities from the Datastore.
\end{description}
