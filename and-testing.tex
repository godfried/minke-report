\subsection{Android}
Testing of our mobile client code was done with unit tests (for testing data and
logic), user testing (for testing the interface) and monkey tools (random
force and instrumentation tests of the interface) .
The unit tests were mainly done to ensure that some of the calculations and
operations done on the Android application performed correctly and allowed
us to see what would happen when the client was faced with unexpected input. We
also made use of some of the InstrumentationTestCase subclasses to test our
Activities and Fragments and see how they would react to various user actions.
User testing was done by:
\begin{itemize}
\item entering extreme values into input areas to ensure that validation and
  input handling was done correctly,
\item repeatedly switching between activities and fragments as well as closing
and opening the application itself to test how it handles sudden
changes in state, and
\item getting users to test the application itself and thereby find bugs or
usability problems in the application.
\end{itemize} 
Monkey tools are some of the application testing tools bundled with the Android
SDK. Specifically they are:
\begin{description}
  \item[monkeyrunner~\cite{runner}] monkeyrunner allows us to write Python
  programs which externally controls an Android device. Using it, we were able
  to install our application, run it, provide it with input, and store
  screenshots of the running application.
\item[Monkey~\cite{monkey}] This program runs on a device and generates
pseudo-random input events like clicks, touches, and gestures, as well as
 system events. Monkey was used to stress-test our application.
\end{description}
lint~\cite{lint} was used to do static analysis on our Android code. lint checks
our project for
\begin{quote}
potential bugs and optimization improvements for correctness,
security, performance, usability, accessibility, and
internationalization.~\cite{lint}
\end{quote} 
Logcat~\cite{logcat} was invaluable as a debugging tool and provides numerous
ways to filter and search through an application's logs logs.

