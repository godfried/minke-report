\subsubsection{Directions Viewer}

The Directions Viewer will function in the same fashion as the GWT Directions Viewer. The library used for this will be the Google Maps API for Android~\cite{amaps}. Directions will be obtained from Google's Directions API via a HTTP request and using JSON to hold the data.
\tikzstyle{small} = [rectangle, fill=white!100, ultra thick, draw=black!80,
            minimum height=3cm, minimum width=7cm,outer sep=0pt] 
\begin{center}
\begin{figure}[H]
\pgfdeclarelayer{background}
\pgfsetlayers{background,main}
\begin{tikzpicture}[node distance=2cm]
\begin{pgfonlayer}{background}
        \node [small] (small) at (0,-3.5){};
    \end{pgfonlayer}
    \node (nodex) [text width=4cm] {};
    \node (shop) [inblock, left of =nodex]
        {
            \textbf{Shopping List Browser}
        };
        \node (product) [inblock, right of= nodex]
        {
            \textbf{Product Browser}
        };
        \node (viewer) [block, below of= nodex]
        {
            \textbf{Directions Viewer}
        };
        \node (nodey) [text width=4cm, below of= viewer]{};
        \node (directions) [block, right of= nodey]
        {
            \textbf{Directions Display}
        };
        \node (map) [block, left of= nodey]
        {
            \textbf{Map Display}
        };
     \path [line] (product) -- (viewer);
     \path [line] (shop) -- (viewer);
     \path [line] (viewer) -- (map);
     \path [line] (viewer) -- (directions);
 \end{tikzpicture}
  \caption{The Directions Viewer.}
 \end{figure}
\end{center}
